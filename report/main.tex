\documentclass[12pt,a4paper]{article}
% This text is inserted in the beginning of all
% LaTex and Tex files I create.
%
% File created: Tue Sep 26 2017
% File name:    report_template.tex
% Path:         /home/name/Classes/AMPIII/Template/
%
% Name
% Sept, 2017
%
% !TEX root = ./main/file.tex
%recommended by fancyhdr package
\setlength{\headheight}{14.49998pt}
\addtolength{\topmargin}{-2.49998pt}

% include a minimal set of useful packages
\usepackage{graphicx}
\usepackage{amsfonts} 
\usepackage{amssymb}
\usepackage{amsmath}
\usepackage[a4paper,margin=4cm]{geometry}
\usepackage{lastpage}
\usepackage{fancyhdr}
\usepackage{float}

% PUT YOUR TITLE AND NAME HERE
\newcommand{\titlestr}{Modelling the Outcome of Tennis Matches \\ Interim Report}
\newcommand{\shorttitlestr}{Modelling Tennis Outcomes}
\newcommand{\authorstr}{S. Gilbert} % INSERT YOUR NAME(S)

\begin{document}
%%%%%%%%%%%%%%%%%%%%%%%%%%%%%%555
% title page
\begin{titlepage}
  \centering

  {\LARGE \titlestr \par}

  \vspace{1cm}
  {\Large \authorstr \par}

  {\bf A1737770.}

  \vspace{1cm}
  \today     % PUT YOUR DATE HERE

  \vspace{2cm}
  Report submitted for
    {\bf Data Science Research Project}
  at the School of Mathematical Sciences,
  University of Adelaide

  \includegraphics[width=0.35\textwidth]{UoA_logo_col_vert.jpg}

  \vspace{2cm}
  \flushleft
  Project Area: {\bf Data Science - Modelling} \\
  Project Supervisor: {\bf Dylan Morris} \\

  \vspace{5mm} {\footnotesize In submitting this work I am indicating
    that I have read the University's Academic Integrity Policy. I
    declare that all material in this assessment is my own work except
    where there is clear acknowledgement and reference to the work of
    others.\par}

  \vspace{5mm} {\footnotesize I give permission for this work
    to be reproduced and submitted to other academic staff for
    educational purposes.\par}

  \vspace{5mm} {\footnotesize I give permission this work
    to be reproduced and provided to future students as an exemplar report.\par}

  \vfill
\end{titlepage}

% put headings on each page
\pagestyle{fancy}
\fancyhf{}
\rhead{\shorttitlestr}
\lhead{\authorstr}
\rfoot{Page \thepage\ of \pageref{LastPage}}
\renewcommand{\headrulewidth}{1pt}

%%%%%%%%%%%%%%%%%%%%%%%%%%%%%%555
% abstract
\begin{abstract}
  Tennis is a multi-billion dollar industry with the outcome of both tournaments
  and individual matches having both enormous monetary ramifications' and impacts
  for player legacy.
  Accurately predicting the outcome of tennis matches underpin
  the tennis bookmaking industry. This leads to profit incentives both on
  the side of the bookmakers and on betters looking to "beat the odds".
  This report investigates various mathematical modelling methods to evaluate
  what is the best performing model when built leveraging past performance
  to determine future outcomes.

  By leveraging open-source datasets of historical tennis data, mathematical models
  can be built to predict the outcome of tennis matches. These models will use a
  subset of the available data initially to form a base performance for these models.
  Future work will be outlined to incorporate more features of the dataset to improve
  the overall performance of the model. This performance was measured using accuracy,
  which is the proportion of guesses that were correct and log loss, which aims to
  measure how confident the model is in its prediction.

  The models investigated ranged from a simple logistics model through to
  several Elo based models. The simpler logistic model use more traditional statistical
  analysis methods for deriving model parameters whereas the Elo models rely on
  predefined hyperparameters to influence the performance of the model.
  It can be observed that the simpler models, ie logistic model or simply predicting
  the higher ranked player outperforms the Elo models when they use unoptimised
  hyperparameters. The log loss however tells a different story, although these
  simpler models have a higher accuracy currently, the Elo models have a substantially
  lower log loss value implying that the future work of tuning the hyperparameters of
  the models should result in greatly improved prediction accuracy. The simpler models
  however cannot be meaningfully modified to improve accuracy. Potential improvements
  to the various models are explored including expanding the parameters used for the
  logistic model and hyperparameter tuning for the various Elo models.
\end{abstract}


\vspace{10mm}
\noindent \hrulefill

%%%%%%%%%%%%%%%%%%%%%%%%%%%%%%555
% main report
\clearpage
\section{Introduction}

Using historical data to predict future events is extremely important across many
areas/fields. A prime instance of this is predicting winners of tennis matches at
a professional level. With sponsorships, player legacy and a multi-billion-dollar
gambling industry relying on the outcomes of each game, being able to predict the
outcome of a game more accurately is of extreme importance. Currently, the
Association of Tennis Professionals (ATP) and the Women's Tennis Association
(WTA) are responsible for ranking players based on performance at applicable
grand slams \cite{nag_tennis_2022}.  Theoretically the highest ranked player should beat
the lower ranked player. However, it can be observed that between 1968 and 2023
in the men's singles competition the higher ranked player wins approximately 66
percent of the time as seen in Figure 1.

\begin{figure}[h!]
  \centering
  \includegraphics[scale=1.5]{images/tennis_rank_base.png}
  \caption{Model Accuracy Using WTA Rankings}
  \label{fig:ranking1}
\end{figure}

Implementing a mathematical model that can perform better translates into higher
returns when matches are bet on or more certainty for companies when trying to
predict forward for offering sponsorship opportunities. There are many approaches
to mathematically model the outcome of tennis matches based on statistical analysis
ranging from simplistic linear/logistical models to complex machine learning
based models. This report investigated a subset of these models and identified
which of these performed the best and what future work will be conducted to
improve the overall performance of these models. Using the naive approach of
always predicting the higher ranked player to win as a baseline the performance
will be compared. Log loss, which measures how close the calculated probability
of an event happening when compared to the actual outcome was used to identify
how confident the various models are in its predictions. This log loss shows that despite
a lower accuracy (how many correct guesses), when it does make correct
predictions it may be more confident than a naive model which will allow for
further tuning to occur to increase the accuracy.
\newline
Based on the results outlined in this report future work will be outlined to
improve the implemented models in order to achieve higher accuracy. As a part
of this future work comparison of the implemented and optimised models against
bookmaker predictions rating will be conducted.

\noindent \hrulefill

\clearpage
\section{Background}

\subsection{Tennis}
Tennis is a two player head-to-head game in which players use a racquet to hit
a ball repeatedly until one of them either hits the ball out of bounds or misses
an in bound shot and the ball subsequently goes out of bounds. Professional tennis
is played within the structure of a tournament which has players playing in rounds,
getting knocked out until two players reach the final. Not all tournaments are
the same, with some leveraging losers brackets and a soccer like point scoring
however these nuances and tournaments are out of scope of this report.

Traditionally these tournaments have a two-sided draw which is designed to have
an even split of the highest ranked players on either side. The intent of this is
to have the higher ranked players play each other in later rounds of the tournament.
The ultimate goal of the tournament organisers is to have the two highest seeds
play each other in the final.

This report focuses on professional men's singles tennis which is overseen by
the ATP. The ATP uses performance at sanctioned events in a rolling 52 week
window to calculate a point value assigned to each player ~\cite{ATPTour2023}.
The rolling window is to remove the chance that a players rank will stay indefinitely
if they stop playing for an extended period of time thus removing "stale" ranks.
These point values are used to determine overall rankings for each professional
player with the player with the highest total points being the "Number 1 Ranked"
player in the world. These rankings are used to allocate seeding at tournaments
and incentivise players to seek as high a seed as they can manage as it means
an easier path to the later rounds. The players best twenty performances at tournaments
contribute to their rating.

\subsection{Existing Models}
\subsubsection{Logistic Model}
The logistic model utilises a transform on predictors/features of the data set to
determine the probability of the higher ranked player winning. For the initial
implementation the difference in rank points will be simply the difference
between the points of the higher ranked player and the lower ranked player. This
model is defined as;

\begin{gather}
  logit(\pi_i) = log(\dfrac{\pi_i}{1-\pi_i}) = \beta_0 + \beta_1 D_i
\end{gather}
with $\beta_0$ and $\beta_1$ being fixed parameters.
The probability can be found by inverting the equation as follows
\begin{gather}
  \pi_i = \dfrac{1}{(1+\exp(-(\beta_0 + \beta_1 D_i)))}
\end{gather}
This model provides a probability based on the difference in rank points and as such the
intercept term, or $\beta_0$, needs to be dropped otherwise it will be added to each
probability which is not appropriate for this problem space.
\subsubsection{Elo Model}
The Elo Rating Algorithm is a commonly used method of ranking players in competitive
games. Some examples of this include official Chess rankings and online Player vs Player (PvP)
video games. Elo measures the probability of a higher ranked player beating a
lower ranked player. Players are assigned an initial rating value which is then updated after each contest.
This value will notionally be set to 1500. A larger difference in score increases the expected probability
of the higher ranked player winning. This probability can be defined as follows
\begin{gather}
  \pi_{i,j}(t) = (1 + 10^{\tfrac{E_j(t)-E_i(t)}{400}})^{-1}
\end{gather}
and updating the players ranking is done as follows
\begin{gather}
  E_i(t+1) = E_i(t) +K_i(W_i(t)-\pi_{i,j}(t))
\end{gather}
Where $W_i(t)$ is an indicator variable for whether the i'th individual won their
t'th match. From the above it can be observed that if the higher ranked player
beats the lower ranked player the rate of change will be lower than if the lower
ranked player beats the higher ranked player.
The K value impacts the rate of change of a player's ranking after a match. Different
approaches for setting this K value will be explored next as it can have a major
impact on the models outputs and therefore its performance.

\subsubsection{K Factor Model}
The most naive approach is to set the K value as a constant and scale the ranking
update based on this constant this K value is defined as follows
\begin{gather}
  K_i(t) = k \quad \forall i
\end{gather}
\subsubsection{Five Thirty Eight Model}
The Five Thirty Eight model introduces some new factors for determining the K
value. Calulating the K value for the Five Thirty Eight model is done as follows.
\begin{gather}
  K_i(t) = \dfrac{\delta}{{(m_i(t) + \nu)}^{\sigma}}
\end{gather}
$\delta$, $\nu$ and $\sigma$ are parameters that need to be tuned to find the
optimal values. This optimisation is out of scope of this initial paper however will be
investigated as future work.
\subsubsection{Elo MMR Model}
A novel approach to calculate a players ranking is a bayesian system resulting in an Elo-MMR
(Massive, Monotonic and Robust) metric \cite{EloMMR2021}. This model uses a gaussian diffusion
to underpin its predictions whilst also provides capability for a logistic performance
model which limits the impact of outlier performances. It differs from other similar
ranking systems such as Microsofts TrueSkill \cite{herbrich_trueskill_2006} by exposing
internal properties and provides mathematical proofs for its optimisation.
Models such as this tend to be vulnerable to volatility farming which is the act of a
player intentionally tanking their rating such that later improved performance will
improve their overall rating due to the models perceived increased player volatility
rating. This models high level implementation can be seen in Figure 2.
\begin{figure}[H]
  \centering
  \includegraphics{images/elommr.PNG}
  \caption{Elo-MMR Definition}
  \label{fig:elommr}
\end{figure}

\subsection{Performance Metrics}
Measuring the performance of the different mathematical models will be done using
two metrics, Accuracy and Log Loss. Accuracy is how many of the matches played
did the model accurately predict the correct outcome. Notionally this metric is
defined as follows.
\begin{gather}
  Accuracy = \dfrac{count(R)}{count(R)+count(W)}
\end{gather}
where $count(R)$ is the number of correct guesses and $count(W)$ is the number
of incorrect guesses.

Log Loss is a measure of how close the prediction probability is to the actual
true/false value. It provides an indication of how confident the output of the
model is in its prediction. Log loss punishes predictions where the predicted
probability is further from the actual outcome more than if the prediction is
closer. Log loss on an individual outcome is defined as follows.
\begin{gather}
  Logloss = -[y \ln p +(1-y)ln(1-p)]
\end{gather}
where $y$ is the actual outcome of the match and $p$ is the prediction probability.
The overall log loss of the model is defined as.
\begin{gather}
  Logloss = -\dfrac{1}{N}\sum_{i=1}^{N}[y_i \ln p_i +(1-y_i)ln(1-p_i)]
\end{gather}
which calculates the average log loss value of all the individual match outcomes
and is the final value used to evaluate the log loss performance of the model.
\clearpage

\section{Method}
Evaluating the previously mentioned models is a multistep process. The first
step is to decide how to "train" the models. There are two approaches that were
considered, splitting the data into a training and testing set. Running the
model over the training set to tune parameters and then test its performance over
the test set to evaluate performance. Traditionally data is split 80-20 into
training and test sets.
The other approach is to, where possible, iterate over each match make a
prediction then update the model. This works well with the Elo models as
updating the point values occurs after the outcome of any given match. For
the logistic model, however, the parameters will be calculated utilising the R studio
logistic model package over the entire dataset. This model will then be used over each
match in the dataset to calculate probabilities of the winner winning. This
approach runs the risk of overfitting as the entire dataset is being used for
both training and testing, however due to the size of the dataset and the minimal
impact an individual game has on the overall model the potential impact of this is
relatively minimal.

\subsection{Dataset}
The data used for this investigation is the Tennis Rankings, Results and Stats open-source
Github repository\cite{sackmann_jeffsackmanntennis_atp_2024} provided by Jeff Sackman.
This dataset contains
statistical information for men's and women's professional tennis including singles and
doubles competitions. Not all information contained in this dataset will be used for all
the models however fields such as playing surface, player hand and tournament location
can be used to test for statistical significance when it comes to predicting outcomes of
tennis matches. Investigating the impact of these features will be investigated in future
work. The dataset is split up based on year and competition type, e.g. men's
singles. For this investigation the focus will be on the men's singles competition.
Dealing with missing information within the dataset will be handled by omitting matches
which don't have all information required for the model to run. Future work will include
alternative approaches to handling models when required information is missing.

\subsection{Logistic Model}
As mentioned above, the slopes for the logistic model are
calculated using R Studio's general linear model (GLM) package. Plotting the
outputs of predicted values against this model are.

\begin{figure}[H]
  \centering
  \includegraphics[scale=0.6]{images/logistic_curve.png}
  \caption{Point Difference v Probability}
  \label{fig:logisticcurve}
\end{figure}
This parameter was found to be $\theta = 0.000565$. As a
result of this the Logistic Model can be defined as follows.
\begin{gather}
  P = -\dfrac{1}{1+e^{-0.000565x}}
\end{gather}
where $x$ is the point difference between the two players. Applying this model
to each match where both players have rank points will generate the results
required to calculate both accuracy and log loss. As well as this,
altering the threshold to only provide a prediction if the calculated probability
is over a certain threshold will be implemented. Testing a range of threshold
values will inform what value provides the best balance of providing predictions
to as many matches as possible whilst maintaining a high level of confidence in the
predictions. This work will be expanded in the future work section of this report.

\subsection{K Factor Model}
Adjusting the value of the K value in the K Factor model will involve iterating
over various values of K to identify any differences in accuracy and log loss
that come with shifting these K values. Initially the value will be set to 5,
once this baseline is established K values between 5 and 50 will be tested in
intervals of 5.

\subsection{Five Thirty Eight Model}
Similar to the K Factor model various values for $\delta$, $\nu$ and $\sigma$
will be iterated over. Since there are three parameters that can be tuned
the search space will need to be reduced when compared to the K Factor model.
Unlike the K Factor model, a brute force approach is most likely inappropriate
and more targeted hyperparameter optimisation will need to be done to identify
the optimal values.
The initial values will be set to $(\delta,\nu,\sigma) = (100,5,0.1)$. Future
work will be to utilise hyperparameter tuning methods to find the optimal values.

\subsection{Elo MMR Model}
Due to the complexity of the Elo MMR model the aim for this report is to
analyse the default implementations effectiveness when applied to the tennis prediction
problem. Similar to the Five Thirty Eight model future work will be conducted
to find optimal hyperparameters and improve performance if applicable. Currently the
implementation being used for this model is a three step process. First, parsing the data
to generate a separate file for each tournament and ranking the performance of each player
that competed. This ranking will be determined by the total number of games won within
that individual tournament. The number of games played is a good measure of performance
for a player as the more games played can be directly tied to how far in a tournament they
progressed. The players will be given a ranking from one to n where
n is the number of players who competed in a specific tournament. Once these separate files
have been generated, these files will be considered a single round in the context of the
Elo-MMR model \cite{EloMMR2021}. The model will then iterate over each tournament and
provide a set of rankings for the players. These rankings will be output as a map of
player names to player ranks at a specific date. Finally the prediction work will be
completed which involves iterating over the initial dataset, using the player ranks at
the date of the match to predict the outcome. If either of the players dont have a rating,
ie it is their first match, a baseline rank of 1500 will be used. Since these results
will be loaded into a map of tournament dates and player rankings at that time, the model
will be capable of using a players current rating value to determine the probability of them
winning the match.


\clearpage
\section{Results}
\subsection{Logistic Model}
Running the logistic model provides the below output
\begin{center}
  \begin{tabular}{||c c c c||}
    \hline
    Correct & Incorrect \\
    \hline\hline
    68084   & 38486     \\
    \hline
  \end{tabular}
\end{center}
For an overall accuracy value of 0.639. The corresponding log loss for this model is 0.481.
With an accuracy lower than simply using the player ATP rank to predict the outcome of a
match the model in this form has definite room for improvement. Since the predictions of
this model in its current form is based solely on the difference between the players rank
points it can be observed that the accuracy of its predictions are similar to using the ATP
rank. In the future work section, capturing more features of the dataset to include in the
logistic model will be done which is an extension not possible with simply using the players
ATP rank.
\subsection{K Factor}
The K Factor model provides one parameter that can be adjusted to alter the models' performance.
This K value impacts both the accuracy and the log loss produced from the model. The table
below shows the accuracy and log loss for each value of K.

\begin{center}
  \begin{tabular}{||c c c c||}
    \hline
    K Value & Accuracy & Log Loss \\
    \hline\hline
    5       & 0.606    & 0.466    \\
    10      & 0.601    & 0.383    \\
    15      & 0.598    & 0.333    \\
    20      & 0.594    & 0.298    \\
    25      & 0.592    & 0.272    \\
    30      & 0.590    & 0.251    \\
    35      & 0.589    & 0.234    \\
    40      & 0.587    & 0.220    \\
    45      & 0.585    & 0.208    \\
    50      & 0.584    & 0.197    \\
    \hline
  \end{tabular}
\end{center}
From the results above it can be observed that as the K value increases, the log loss decreases.
This, however, corresponds with a decline in the models overall accuracy. As this model
does not consider the past performance of the players when updating their corresponding
ranks potential improvements to this model are minimal. The following plot tracks the performance
of Roger Federer across his long career.

\begin{figure}[H]
  \centering
  \includegraphics[scale=0.8]{images/federer_k_factor.png}
  \caption{K Factor Federer Rank Over Time}
  \label{fig:federer-kfactor}
\end{figure}

It can be observed the model captures the trajectory of his career quite well with a slow
start corresponding to his poor performances at the start of his career, an extended peak
during the middle of his career coinciding with his "prime" and then a slight drop off
towards the end of his career before his retirement.

\subsection{Five Thirty Eight Model}
As discussed in the Method section of this report the Five Thirty Eight model relies on
three hyperparameters to define the model. The values $(\delta,\nu,\sigma) = (100,5,0.1)$
are known to be suboptimal, and the following results are just to demonstrate in initial
implementation.
Running the Five Thirty Eight model with these parameters provides an accuracy value of
0.577 and an overall log loss of 0.167. This shows similar performance to the K Factor model
with a lower corresponding log loss value implying a higher confidence in the models predictions.
Hyperparameter optimisation should improve the performance of this model further and this
will be expanded further in the future work section of this report. Similar to the K Factor
result, the career ranking of Roger Federer has been plotted below.

\begin{figure}[H]
  \centering
  \includegraphics[scale=0.8]{images/federer_538.png}
  \caption{Five Thirty Eight Federer Rank Over Time}
  \label{fig:federer-538}
\end{figure}

This plot looks almost identical to the K Factor ranking for Federer which explains why
accuracy of the models is quite similar. The Five Thirty Eight model has Federer having a
higher peak rating with a marginally higher rate of change when outlier performances occur.

\subsection{Elo-MMR Model}
A baseline implementation of the Elo-MMR model has been implemented to set a baseline for
future work to build on. The model produced an accuracy of 0.635 with a log loss of 0.533.
This model outperformed the other Elo models in accuracy with a significantly worse log loss.
Due to the complexity of the model there are many reasons why the model lacks confidence
in its predictions. By plotting the performance of a known high performing player, Roger Federer,
deficiencies in the model can be observed.

\begin{figure}[H]
  \includegraphics[scale=0.6]{images/federer.png}
  \caption{Elo-MMR Federer Rank Over Time}
  \label{fig:federer-elo-mmr}
\end{figure}

From the above plot, it can be observed that the model is extremely punitive for a player
if their performance early in their career is poor and doesn't appropriately rate higher
performances to compensate. Future work will include investigating methods of considering
the difficulty in reaching higher rounds of a tournament and updating the players rank to
more accurately reflect that.

\vspace{10mm}
\noindent \hrulefill

\clearpage
\section{Discussion}
From the results of the initial implementations of these models there is some key
takeaways. The basic logistic model currently has the highest accuracy meaning that it
makes the most correct predictions. This is because the model in its current form is very
similar to simply using the ATP rank for the prediction as that value is based on the players
current rank point totals. A higher point total corresponds to a higher ATP rank resulting
in essentially the same prediction. In its current state this logistic model currently only
considers a single parameter, the players rank points, however there is opportunity to
extend this model to include other parameters present in the dataset

The current implementations of the various Elo models captures a similar trajectory in a
players career. This can be observed by overlaying the two Elo models which are updated
on a game by game basis, K Factor and Five Thirty Eight. The Elo-MMR model is calculated
on a tournament basis so cannot be overlaid simply on the same plot.
\begin{figure}[H]
  \includegraphics[scale=0.8]{images/federer_comparison.png}
  \caption{K Factor vs Five Thirty Eight}
  \label{fig:federer-k-538}
\end{figure}

From this plot it can be observed that the Five Thirty Eight Model varies more than the
K Factor model as a result of individual matches. This is because it considers the players
history when determining a K value for the rank update. Outlier performances have a greater
impact on an individual update which results in a lower rating being assigned early in the
career with poorer performances occurring. Conversely, as Federers performance increases,
the improvement is picked up earlier in the Five Thirty Eight model when compared to the
K Factor model. This results in the model reaching a steady state that accurately reflects
Federers skill earlier than the K Factor model which when optimal parameters are identified
should result in a higher overall accuracy.

The Elo-MMR model accurately captures the trajectory of Federers career as shown in Figure 6.
This model seems to not appropriately reward a player for excellent performance in a tournament
and is extremely confident in its initial rating of a player. From the models calculated
ratings, this results in a large percentage of the players having ratings that is around
the same values. As a result the models predictions is often based on very similar player
ratings. By more accurately capturing performance in tournaments and incorporating that
in to rating updates there will be more variance in player ratings resulting in more
accurate predictions. The current approach for rating performance for a specific tournament
is to assign a rank based on the round reached within the tournament. This results in an
accurate rating for the first and second best performing players in the tournament as they
are the ones who made it to the final. The issue arises when ranking players who get eliminated
in earlier rounds. The losers of the semi-finals will be tied for the third-best performance
in the tournament. This becomes a problem when considering earlier rounds as the number of
players with the same performance increases exponentially. Alternative approaches for more
accurately capturing a players performance at a tournament will be expanded in future work.

Currently, all the models in their current states have significantly lowered prediction accuracy
than the simple approach of making a prediction solely based on the players ATP rank.
The true value in the above work is that these models have much more potential for extension
or modification to improve their accuracy. As mentioned, the logistic model can be
extended with more parameters, the Elo models can have hyperparameters tuned and since
these models are probability based the threshold at which a prediction is made can be
changed.

\subsection{Future Work}
Extending the investigated models will be the focus for the next part of this research
project. Starting with the logistics model the natural extension of the current model
is to investigate additional parameters to include. Utilising the R logistics model
package, both additional parameters and interaction terms will be tested for statistical
significance to the match outcomes. Incorporating playing surface and playing hand will
be investigated first as there is some evidence that it can have an impact on match
outcome \cite{loffing_left-handedness_2012}.

The basic K Factor Elo model will not be investigated any further as the various values
of K and its impact on model performance have already been investigated above. The Five
Thirty Eight model leverages three independent hyperparameters to calculate the change
in ranking after each contest. The search space for brute forcing the optimal values of
these hyperparameters is infeasible. There are many methods of optimisation for
hyperparameter search \cite{claesen_hyperparameter_2015} and investigating the
most appropriate one for optimising the Elo model parameters will form the basis of
future work for improving the Five Thirty Eight model.

From the results of the Elo-MMR model there are multiple opportunities for optimisation.
By plotting the score of a single player known to have a higher than average performance
across a long period of time, Roger Federer, deficiencies in the model performance can be
identified. Initial observations indicate that the model becomes heavily confident in
its predictions based on player performance early in their career. This results in reducing
the rate of change of a player ranking for matches later in their career. The problem
with this is player performance early in a career is most likely lower than once they
get more experienced in professional play. Currently, the Elo-MMR model is updated for each
player after each tournament. To combat the issue of overconfidence in the model based
on early career performance, the players initial ranking will be produced from a set of their
first n matches or the first n tournaments. Various values of this initial bucket size
will be investigated to
optimise between model confidence and maximising the number of matches with a player ranking.
An investigation will also be conducted to determine the models' performance if instead of
updating the ratings on a tournament basis instead it is done on a game by game basis
like the other Elo models. The risk of implementing the model in this way is that the
identified issue of over-confidence based on early career performance could be increased.
The current implementation of the model defaults to the ATP rankings of the players if
both players dont have a valid Elo-MMR ranking. By increasing the required number of games
played by a player before calculating a Elo-MMR rank the number of games that will default
to the ATP rankings will increase. This potentially reduces the accuracy of the model and
needs to be minimised. The other optimisation that will be investigated is instead of updating the Elo-MMR after
every tournament, the players rank will be updated after n amount of tournaments.
This will result in a higher confidence in the update of the model ranking.

The current method of determining whether any of these Elo models have produced a correct
prediction is by determining if the probability of player i beating player j is greater
than 0.5. Logically this is a good starting point as it implies that theres is a greater
than 50\% chance of a player winning. Setting this probability threshold to different values
will be investigated further to identify any improvements in the models prediction accuracy.
With a higher threshold, it is expected that the model will make predictions on less
matches as when players have similar Elo scores the model will not make a prediction.
Balancing the improvement to model accuracy versus the reduced number of matches in which
a prediction is made will be completed and discussed.

Comparing model performance against open source metrics to rate performance of the various
models will be included in future work. Investigations into these various metrics will be
conducted. The Five Thirty Eight model with optimal parameters has been found to provide
predictions with approximately 75\% accuracy \cite{kovalchik_searching_2016}. This is just
one of the metrics that will be compared and will be used to set a baseline of which to


\clearpage

\section{Conclusion}
Performance of these models have been quantified by measuring both
model accuracy and the log loss of the models predictions. There are many mathematical models that can be applied to the tennis prediction problem.
These models range in both complexity and performance as such robust performance testing
has been conducted to identify which models perform the best. Performance of these models
have been quantified by measuring both model accuracy and the log loss of the models
predictions. An extensive open-source dataset of historical tennis data was leveraged to
both build and test these models. Currently a small subset of the available data is being
used to build these models. Future work will include incorporating a greater number of
features within the dataset. Capturing more features of the available data and conducting
statistical analysis on its impact on match outcome will result in more accurate predictions
with a higher confidence as measured by its corresponding log loss value.

The logistic model performs well with just a basic classifier based on a
players ATP rank. The log loss of the logistic model, is higher than other models
investigated in this report. This implies that the model,
confidence in its predictions is lower than alternatives such as the Elo models.
Attempting to address this issue will be conducted as a part of future work by
extending the parameters used to build the logistic model to increase the models
confidence in its predictions.

Various Elo models were implemented and their respective performances compared. These Elo
models have similar performance to the basic logistics model however they demonstrate a
lower log loss value. This demonstrates that these models are more confident in their
predictions so if they can be modified to increase the accuracy the model will outperform
the basic logistic model. Differences in the ratings' of an individual player
generated by the individual models shows that they all succeed in capturing the overall
trajectory of a players career. The different Elo models update the player ratings at
different rates based on performance. Having the model accurately update these ratings
is the key to maximising accuracy whilst minimising log loss. The more complex Elo models
such as the Five Thirty Eight and the Elo-MMR provide more mechanisms for tuning than the
simpler ones such as the K Factor model. This comes at the cost of model complexity as
incorrectly configured models can have a massive impact on its performance. Balancing
the need to optimise the models to solve the given problem with model complexity will
be investigated in future work.

\clearpage

\bibliographystyle{plain}
\bibliography{bib_file}

\vspace{10mm}
\noindent \hrulefill

\end{document}


